% !TeX root = ../main.tex
% Add the above to each chapter to make compiling the PDF easier in some editors.

\chapter{Introduction}\label{chapter:introduction}
Test ref to Listing~\ref{lst:1}. Test ref to Listing~\ref{lst:3}
Test ref to Gen6D \cite{liu2023gen6d}
\section{Problem Formulation}
\subsection{The settings}
\subsection{The goal}
\section{The work environment: Scitas Izar}
Test to refer to the video from 3B1B: \cite{3b1b-1}.

\begin{lstlisting}[style=bashstyle, caption={Bash script \texttt{execute.sh} to run a machine learning model on Scitas Izar EPFL. While the overall structure remains consistent, this script is specific to Gen6D's architecture, further discussed later.}]
#!/bin/bash
#SBATCH --chdir /scratch/izar/jchavero
#SBATCH --partition=gpu
#SBATCH --qos=gpu_free
#SBATCH --gres=gpu:2
#SBATCH --nodes=1
#SBATCH --ntasks-per-node=1
#SBATCH --cpus-per-task=1
#SBATCH --mem 16G

echo STARTING AT `date`

echo "Loading modules"
module load gcc openmpi py-torch py-torchvision cuda

echo "Launching the virtual environment"
source ~/opt/izar1/venv-gcc/bin/activate

echo "Navigating to the directory and executing the task"
cd ~/Gen6D                                    
python eval.py --cfg configs/gen6d_pretrain.yaml --object_name spacecraft/hubble

echo FINISHED AT `date`
\end{lstlisting}

\noindent Then to run the script we use the following command:
\begin{lstlisting}[style=bashstyle, caption=Linux command to run the bash script.]
	$ sbatch execute.sh
\end{lstlisting}