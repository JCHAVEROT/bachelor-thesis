% !TeX root = ../main.tex
% Add the above to each chapter to make compiling the PDF easier in some editors.

\chapter{Implementation of the model}\label{chapter:implementation_of_the_model}

\section{Data Loader}
abstract base classes (ABC)
each and every abstract method

\section{From Quaternions to Rotation Matrices}
\pagestyle{fancy}
\fancyhf{}
\fancyhead[C]{\small\textsc{3.2. From Quaternions to Rotation Matrices}}
\renewcommand{\footrulewidth}{0.4pt} % adds a line above the footer
\fancyfoot[C]{\small\thepage} % center the page number in the footer

In Gen6D, the model represents the ground truth and estimated poses using the format ${\bm{\mathrm{P}}}=(\bm{\mathrm{R}},\,\bm{\mathrm{t}})$. Here, $\bm{\mathrm{R}}$ denotes the rotation matrix and $\bm{\mathrm{t}}$ is the translation vector.
The thing is, in the \textsc{SpaceCraft} dataset, the poses have the format ${\bm{\mathrm{P}}}=(\underline{\bm{\mathrm{q}}},\,\bm{\mathrm{t}})$, where $\underline{\bm{\mathrm{q}}}$ is a \textit{quaternion}. We use quaternions for three-dimensional rotation calculations because they offer several benefits over rotation matrices. Notably, quaternions are more compact, requiring only four elements to be stored compared to nine for a matrix. Additionally, they are more efficient when composing rotations thanks to their algebraic properties.

Despite the benefits of quaternions mentioned earlier, other datasets frequently represent poses using a combination of rotation matrices and translation vectors. Therefore, to align with Gen6D's pose format, this section will focus on converting quaternions into rotation matrices.

\bigbreak 

Quaternions were first introduced by the Irish mathematician W. R. Hamilton in 1843 as an extension of the complex numbers \cite{Hamilton1866}. In the first place, we provide the definition of a \textit{quaternion}: it is the sum of a scalar $\mathrm{q}_0$ and a vector $\bm{\mathrm{q}}=(\mathrm{q}_1, \mathrm{q}_2, \mathrm{q}_3)$, that is,
\begin{center}
	$\underline{\bm{\mathrm{q}}} \stackrel{\text{def.}}{=} \mathrm{q}_0 + \bm{\mathrm{q}} = \mathrm{q}_0 + \mathrm{q}_1\bm{i} + \mathrm{q}_2\bm{j} + \mathrm{q}_3\bm{k}$.
\end{center}

\noindent In the above, $\bm{i}$, $\bm{j}$ and $\bm{k}$ denote the three unit vectors of the canonical basis for 
the set of all ordered triples of real numbers $\mathbb{R}^3$. The set of quaternions is denoted by the 4-space $\mathbb{H}$.

\bigbreak 
The quaternion addition is component-wise. Regarding the product of two quaternions, it is essential to first outline the foundational rule established by \mbox{Hamilton}:
\begin{align*}
	\bm{i}^2 &= \bm{j}^2 = \bm{k}^2 = \bm{ijk}= -1.
\end{align*}

\noindent We derive the following multiplication table:

\begin{center}
\setlength{\arrayrulewidth}{1pt}
\renewcommand{\arraystretch}{1.2}
\begin{tabular}{|m{1.5em}||m{1.5em}|m{1.5em}|m{1.5em}|m{1.5em}|}
    \hline
    \rowcolor{gray!25}
    \cellcolor{gray!60}$\times$ & \centering $1$ & \centering $\bm{i}$ & \centering $\bm{j}$ & \centering $\bm{k}$ \tabularnewline
    \hline\hline
    \cellcolor{gray!25}$1$ & \centering $1$ & \centering $\bm{i}$ & \centering $\bm{j}$ & \centering $\bm{k}$ \tabularnewline
    \hline
    \cellcolor{gray!25}$\bm{i}$ & \centering $\bm{i}$ & \centering $-1$ & \centering $\bm{k}$ & \centering $-\bm{j}$ \tabularnewline
    \hline
    \cellcolor{gray!25}$\bm{j}$ & \centering $\bm{j}$ & \centering $-\bm{k}$ & \centering $-1$ & \centering $\bm{i}$ \tabularnewline
    \hline
    \cellcolor{gray!25}$\bm{k}$ & \centering $\bm{k}$ & \centering $\bm{j}$ & \centering $-\bm{i}$ & \centering $-1$ \tabularnewline
    \hline
\end{tabular}
\end{center}

\bigbreak 

\noindent Let $(\underline{\bm{\mathrm{p}}}, \underline{\bm{\mathrm{q}}})\in\mathbb{H}^2$, we are now able to present the multiplication of $\underline{\bm{\mathrm{p}}}$ and $\underline{\bm{\mathrm{q}}}$:
\begin{align*}
    \underline{\bm{\mathrm{p}}}\underline{\bm{\mathrm{q}}} &= \underbrace{\mathrm{p}_0\mathrm{q}_0 - \bm{\mathrm{p}} \cdot \bm{\mathrm{q}}}_{\text{scalar part}} + \underbrace{\mathrm{p}_0\bm{\mathrm{q}} + \mathrm{q}_0\bm{\mathrm{p}} + \bm{\mathrm{p}} \times \bm{\mathrm{q}}}_{\text{vector part}}.
\end{align*}

\noindent In the preceding, we recall that the binary operation $\mathbb{R}^3\times\mathbb{R}^3\rightarrow\mathbb{R}$ : $(\bm{\mathrm{p}},\bm{\mathrm{q}})\mapsto\bm{\mathrm{p}}\cdot\bm{\mathrm{q}}\stackrel{\text{def.}}{=}\bm{\mathrm{p}}^\intercal\bm{\mathrm{q}}$ is the \textit{dot product}, and the other
\begin{align*}
	\mathbb{R}^3\times\mathbb{R}^3\rightarrow\mathbb{R}^3 : (\bm{\mathrm{p}},\bm{\mathrm{q}})\mapsto\bm{\mathrm{p}}\times\bm{\mathrm{q}}\stackrel{\text{def.}}{=}\begin{vmatrix} \bm{i} & \bm{j} & \bm{k} \\ \mathrm{p}_1 & \mathrm{p}_2 & \mathrm{p}_3 \\
		\mathrm{q}_1 & \mathrm{q}_2 & \mathrm{q}_3 \end{vmatrix}
		=[\bm{\mathrm{p}}]_{_{_\times}}\bm{\mathrm{q}}
\end{align*}

\noindent is the \textit{cross product}, and $[\bm{\mathrm{p}}]_{_{_\times}}$denotes the transformation matrix that when multiplied from the right with a vector $\bm{\mathrm{q}}$ gives $\bm{\mathrm{p}}\times\bm{\mathrm{q}}$.
\bigbreak 
Several additional definitions are essential before we can begin the conversion of quaternions to rotation matrices. 


\noindent Let $\underline{\bm{\mathrm{q}}}=\mathrm{q}_0 + \bm{\mathrm{q}}$ be a quaternion. The \textit{complex conjugate} of $\underline{\bm{\mathrm{q}}}$, denoted by $\underline{\bm{\mathrm{q}}}^\star$, is given by the map
\begin{align*}
	\mathbb{H}\rightarrow\mathbb{H}:\underline{\bm{\mathrm{q}}}\mapsto\underline{\bm{\mathrm{q}}}^\star\stackrel{\text{def.}}{=}\mathrm{q}_0 - \bm{\mathrm{q}}.
\end{align*} 

\noindent The \textit{norm} of a quaternion $\underline{\bm{\mathrm{q}}}$, denoted $|\underline{\bm{\mathrm{q}}}|$, is the distance obtained from the map 
\begin{align*}
	\mathbb{H}\rightarrow\mathbb{R}^+:\underline{\bm{\mathrm{q}}}\mapsto|\underline{\bm{\mathrm{q}}}|\stackrel{\text{def.}}{=}\sqrt{\underline{\bm{\mathrm{q}}}^\star\underline{\bm{\mathrm{q}}}}.
\end{align*} 
Note that a quaternion whose norm is 1 is referred to as a \textit{unit quaternion}. The \textit{reciprocal} of a quaternion is defined as the map
\begin{align*}
	\mathbb{H}^*\rightarrow\mathbb{H}^*:\underline{\bm{\mathrm{q}}}\mapsto\underline{\bm{\mathrm{q}}}^{-1}\stackrel{\text{def.}}{=}\frac{\underline{\bm{\mathrm{q}}}^\star}{|\underline{\bm{\mathrm{q}}}|^2}.
\end{align*} 
We observe that if $\underline{\bm{\mathrm{q}}}$ is a unit quaternion, we simply have $\underline{\bm{\mathrm{q}}}^{-1}=\underline{\bm{\mathrm{q}}}^\star$. Furthermore, the subsequent proposition is accepted: if $\underline{\bm{\mathrm{q}}}$ is a unit quaternion, there exists a unique $\theta\in[0,2\pi]$ such that
\begin{align*}
	\underline{\bm{\mathrm{q}}} = \mathrm{q}_0 + \bm{\mathrm{q}}=\cos\frac{\theta}{2}+\bm{\mathrm{u}}\,\sin\frac{\theta}{2}\,,
\end{align*}
where the unit vector $\bm{\mathrm{u}}$ is defined as $\bm{\mathrm{u}}\stackrel{\text{def.}}{=}\frac{\bm{\mathrm{q}}}{|\bm{\mathrm{q}}|}$.

\setlength{\belowdisplayskip}{0.15cm}

\paragraph{Quaternion Rotation Operator} Let $\underline{\bm{\mathrm{q}}}\in\mathbb{H}$ be a unit quaternion, and let $\bm{\mathrm{v}}\in\mathbb{R}^3$ be a vector. The action of the $L_{\underline{\bm{\mathrm{q}}}}$ function 
\begin{align*}
	\mathbb{R}^3\rightarrow\mathbb{R}^3:\bm{\mathrm{v}}\mapsto L_{\underline{\bm{\mathrm{q}}}}(\bm{\mathrm{v}}) \stackrel{\text{def.}}{=} \underline{\bm{\mathrm{q}}}\bm{\mathrm{v}}\underline{\bm{\mathrm{q}}}^\star
\end{align*}
on $\bm{\mathrm{v}}$ is equivalent to a rotation of the vector through an angle $\theta$ about the axis of rotation $\bm{\mathrm{u}}$.

\setlength{\belowdisplayskip}{0.3cm}

\bigskip At last, we can proceed with our conversion task: our aim is to find a $3\times3$ rotation matrix $\bm{\mathrm{R}}$, such that
\begin{align*}
\left\{
    \begin{aligned}
    	L_{\bm{\mathrm{R}}}(\bm{\mathrm{v}}) &\stackrel{\text{def.}}{=} \bm{\mathrm{R}}\bm{\mathrm{v}} \\
        L_{\bm{\mathrm{R}}}(\bm{\mathrm{v}}) &= L_{\underline{\bm{\mathrm{q}}}}(\bm{\mathrm{v}}),
    \end{aligned}
\right.
\end{align*}

which means we wish to obtain an expression for $\bm{\mathrm{R}}$ by manipulating $L_{\underline{\bm{\mathrm{q}}}}\,$, utilizing principles of linear algebra and vector calculus. We consider the vector $\bm{\mathrm{v}}$ as a quaternion with a zero scalar component:


\begin{align*}
    L_{\underline{\bm{\mathrm{q}}}}(\bm{\mathrm{v}}) &= \underline{\bm{\mathrm{q}}}\bm{\mathrm{v}}\underline{\bm{\mathrm{q}}}^\star \\
    &= (\mathrm{q}_0+\bm{\mathrm{q}})(0+\bm{\mathrm{v}})(\mathrm{q}_0-\bm{\mathrm{q}}) \\
    &= (\underbrace{-\bm{\mathrm{q}}\cdot\bm{\mathrm{v}}}_{\text{scalar part}} + \underbrace{\mathrm{q}_0\bm{\mathrm{v}}+\bm{\mathrm{q}}\times\bm{\mathrm{v}}}_{\text{vector part}})(\mathrm{q}_0-\bm{\mathrm{q}}) \\
    &= \mathrm{q}_0(-\bm{\mathrm{q}}\cdot\bm{\mathrm{v}}) - (\mathrm{q}_0\bm{\mathrm{v}}+\bm{\mathrm{q}}\times\bm{\mathrm{v}})\cdot(-\bm{\mathrm{q}}) \\
    &\quad + (-\bm{\mathrm{q}}\cdot\bm{\mathrm{v}})(-\bm{\mathrm{q}}) + \mathrm{q}_0(\mathrm{q}_0\bm{\mathrm{v}}+\bm{\mathrm{q}}\times\bm{\mathrm{v}}) \\
    &\quad + (\mathrm{q}_0\bm{\mathrm{v}}+\bm{\mathrm{q}}\times\bm{\mathrm{v}})\times(-\bm{\mathrm{q}}) \\
    &= \underbrace{\cancel{-\mathrm{q}_0(\bm{\mathrm{q}}\cdot\bm{\mathrm{v}}) + \mathrm{q}_0(\bm{\mathrm{q}}\cdot\bm{\mathrm{v}})}}_{\text{scalar part}} \\
    &\quad + \underbrace{\bm{\mathrm{q}}\,(\bm{\mathrm{q}}\cdot\bm{\mathrm{v}}) + {\mathrm{q}_0}^2\,\bm{\mathrm{v}} + \mathrm{q}_0(\bm{\mathrm{q}}\times\bm{\mathrm{v}}) + \bm{\mathrm{q}}\times(\mathrm{q}_0\bm{\mathrm{v}}+\bm{\mathrm{q}}\times\bm{\mathrm{v}})}_{\text{vector part}} \\
    &= \bm{\mathrm{q}}\,(\bm{\mathrm{q}}^\intercal\bm{\mathrm{v}}) + {\mathrm{q}_0}^2\,\bm{\mathrm{v}} + \mathrm{q}_0(\bm{\mathrm{q}}\times\bm{\mathrm{v}}) + \bm{\mathrm{q}}\times(\mathrm{q}_0\bm{\mathrm{v}}+\bm{\mathrm{q}}\times\bm{\mathrm{v}}) \\
    &= (\bm{\mathrm{q}}\otimes\bm{\mathrm{q}} + {\mathrm{q}_0}^2\,\bm{\mathrm{I}}_{3\times3} + 2\,\mathrm{q}_0\,[\bm{\mathrm{q}}]_{_{_\times}}\!\! + [\bm{\mathrm{q}}]_{_{_\times}}^2)\,\,\bm{\mathrm{v}}
\end{align*}

\noindent In the above, $\otimes$ stands for the \textit{outer product} and $\bm{\mathrm{I}}_{3\times3}$ is the \textit{identity matrix}. 

\noindent Since $L_{\bm{\mathrm{R}}}(\bm{\mathrm{v}}) = L_{\underline{\bm{\mathrm{q}}}}(\bm{\mathrm{v}})$, we can identify $\bm{\mathrm{R}}$ as  $(\bm{\mathrm{q}}\otimes\bm{\mathrm{q}} + {\mathrm{q}_0}^2\,\bm{\mathrm{I}}_{3\times3} + 2\,\mathrm{q}_0\,[\bm{\mathrm{q}}]_{_{_\times}}\!\! + [\bm{\mathrm{q}}]_{_{_\times}}^2)$. We develop and simplify $\bm{\mathrm{R}}$ to find its final expression:

\begin{align*}
	\bm{\mathrm{R}} &= \bm{\mathrm{q}}\otimes\bm{\mathrm{q}} + {\mathrm{q}_0}^2\,\bm{\mathrm{I}}_{3\times3} + 2\,\mathrm{q}_0\,[\bm{\mathrm{q}}]_{_{_\times}}\!\! + [\bm{\mathrm{q}}]_{_{_\times}}^2 \\
	&= \begin{bmatrix}
    	{\mathrm{q}_1}^2 & \mathrm{q}_1\mathrm{q}_2 & \mathrm{q}_1\mathrm{q}_3 \\
    	\mathrm{q}_2\mathrm{q}_1 & {\mathrm{q}_2}^2 & \mathrm{q}_2\mathrm{q}_3 \\
    	\mathrm{q}_3\mathrm{q}_1 & \mathrm{q}_3\mathrm{q}_2 & {\mathrm{q}_3}^2
		\end{bmatrix} 
		+ {\mathrm{q}_0}^2
		\begin{bmatrix}
    	1 & 0 & 0 \\
    	0 & 1 & 0 \\
    	0 & 0 & 1
		\end{bmatrix}
		+
		2\,\mathrm{q}_0 \begin{bmatrix}
    	0 & -\mathrm{q}_3 & \mathrm{q}_2 \\
    	\mathrm{q}_3 & 0 & -\mathrm{q}_1 \\
    	-\mathrm{q}_2 & \mathrm{q}_1 & 0
		\end{bmatrix} \\
		&\quad+
		\begin{bmatrix}
    	0 & -\mathrm{q}_3 & \mathrm{q}_2 \\
    	\mathrm{q}_3 & 0 & -\mathrm{q}_1 \\
    	-\mathrm{q}_2 & \mathrm{q}_1 & 0
		\end{bmatrix}
		\begin{bmatrix}
    	0 & -\mathrm{q}_3 & \mathrm{q}_2 \\
    	\mathrm{q}_3 & 0 & -\mathrm{q}_1 \\
    	-\mathrm{q}_2 & \mathrm{q}_1 & 0
		\end{bmatrix} \\\\
		&=2
		\begin{bmatrix}
    	({\mathrm{q}_0}^2+{\mathrm{q}_1}^2)-\frac{1}{2} & \mathrm{q}_1\mathrm{q}_2-\mathrm{q}_0\mathrm{q}_3 & \mathrm{q}_1\mathrm{q}_3+\mathrm{q}_0\mathrm{q}_2 \\
    	\mathrm{q}_1\mathrm{q}_2+\mathrm{q}_0\mathrm{q}_3 & ({\mathrm{q}_0}^2+{\mathrm{q}_2}^2)-\frac{1}{2} & \mathrm{q}_2\mathrm{q}_3-\mathrm{q}_0\mathrm{q}_1 \\
    	\mathrm{q}_1\mathrm{q}_3-\mathrm{q}_0\mathrm{q}_2 & \mathrm{q}_2\mathrm{q}_3+\mathrm{q}_0\mathrm{q}_1 & ({\mathrm{q}_0}^2+{\mathrm{q}_3}^2)-\frac{1}{2}
		\end{bmatrix}
\end{align*} 

\bigskip

\noindent This completes our derivation of the rotation matrix $\bm{\mathrm{R}}$. The Python code can be found in Listing~\ref{lst:3}.
