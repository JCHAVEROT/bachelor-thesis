\chapter{Appendix}

\begin{lstlisting}[language=Python, label=lst:1, caption=Python script \texttt{format.py} to randomly generate the training set and the test set based on a specified probability. Should be run from Gen6D's root folder.]
"""
Author:      Jeremy Chaverot
Date:        November 29, 2023
Description: Create the files val.txt, train.txt and test.txt according to a test percentage
"""

import os
import sys
import random


if __name__ == "__main__":

	# Check if the correct number of arguments is provided
    if len(sys.argv) != 3:
        print("Usage: python format.py <object_name> <test_percentage>")
        sys.exit(1)

    object = sys.argv[1]
    test_percentage = float(sys.argv[2])

    if (test_percentage < 0 or 1 < test_percentage):
        print("Wrong value for the variable <test_percentage>. Should be between 0 and 1 included.")
        sys.exit(1)

    # Get a list of all files in the folder
    all_files = os.listdir(f'data/SpaceCraft/{object}/images')

    # Filter the list to include only image files and exclude MacOS temporary files
    image_files = [file for file in all_files if file.lower().endswith(('.jpg')) and not file.startswith('._')]

    # Get the number of images in the folder
    num_images = len(image_files)

    # Iterate through each image and apply the transformation
    with open(f'data/SpaceCraft/{object}/train.txt', 'w') as train, open(f'data/SpaceCraft/{object}/test.txt', 'w') as test:
        for image_file in image_files:
            rand = random.random()
            image_path = 'SpaceCraft/hubble/images/' + image_file
            if (rand < test_percentage):
                test.write(image_path + '\n')
            else: train.write(image_path + '\n')

    print(f"Done splitting {num_images} images in train.txt and test.txt")
\end{lstlisting}

%\cleardoublepage{}
\bigskip

\begin{lstlisting}[language=Python, label=lst:2, caption=Python script \texttt{to\_jpg.py} to transform every images of a specified folder into \texttt{jpg} format.]
"""
Author:      Jeremy Chaverot
Date:        November 20, 2023
Description: Transform every images of a folder into jpg format.
"""

import os
import sys
from PIL import Image


def transform_image(image_path):
    img = Image.open(image_path)
    new_image_path = image_path.split('.')[0] + '.jpg'
    img.save(new_image_path)


if __name__ == "__main__":

	# Check if the correct number of arguments is provided
    if len(sys.argv) != 2:
        print("Usage: python to_jpg.py </path/to/your/images>")
        sys.exit(1)

    folder_path = sys.argv[1]

    # Get a list of all files in the folder
    all_files = os.listdir(folder_path)
    
    # Filter the list to include only image files and exclude MacOS temporary files
    image_files = [file for file in all_files if file.lower().endswith(('.png', '.jpg', '.jpeg', '.gif', '.bmp')) and not file.startswith('._')]
    
    # Get the number of images in the folder
    num_images = len(image_files)

    # Iterate through each image and apply the transformation
    for image_file in image_files:
        image_path = os.path.join(folder_path, image_file)
        transform_image(image_path)
        os.remove(image_path)

    print(f"Number of images transformed into .jpg: {num_images}")
\end{lstlisting}


%\cleardoublepage{}
\bigskip

\begin{lstlisting}[language=Python, label=lst:3, caption=Python script \texttt{quaternion\_to\_matrix.py} to transform a txt file with quaternions and the translation vector into multiple npy files containing the rotation matrix augmented with the translation vector.]
"""
Author:      Jeremy Chaverot
Date:        November 20, 2023
Description: Transform a txt file with quaternions and the translation vector into multiple npy files containing the rotation matrix augmented with the translation vector.
"""

import numpy as np
import sys
import os


def quaternion_to_matrix(Q, translation):
    """
        Covert a quaternion and translation into a full three-dimensional augmented rotation matrix.

        Input
        :param Q: A 4 element array representing the quaternion (qw, qx, qy, qz).
        :param translation: A 3 element array representing the translation (x, y, z).

        Output
        :return: A 3x4 element matrix representing the full 3D rotation matrix with
                 translation. This rotation matrix converts a point in the local
                 reference frame to a point in the global reference frame.
    """

    # Extract the values from Q
    qw = Q[0]
    qx = Q[1]
    qy = Q[2]
    qz = Q[3]

    # Extract the values from the translation vector
    x = translation[0]
    y = translation[1]
    z = translation[2]

    # First row of the rotation matrix
    r00 = 2 * (qw * qw + qx * qx) - 1
    r01 = 2 * (qx * qy - qw * qz)
    r02 = 2 * (qx * qz + qw * qy)

    # Second row of the rotation matrix
    r10 = 2 * (qx * qy + qw * qz)
    r11 = 2 * (qw * qw + qy * qy) - 1
    r12 = 2 * (qy * qz - qw * qx)

    # Third row of the rotation matrix
    r20 = 2 * (qx * qz - qw * qy)
    r21 = 2 * (qy * qz + qw * qx)
    r22 = 2 * (qw * qw + qz * qz) - 1

    # 3x3 rotation matrix
    rot_matrix_augm = np.array([[r00, r01, r02, x],
                                [r10, r11, r12, y],
                                [r20, r21, r22, z]])

    return rot_matrix_augm


if __name__ == "__main__":

    # Check if the correct number of arguments is provided
    if len(sys.argv) != 3:
        print("Usage: python quaternion_to_matrix.py </path/to/your/text/file> </path/to/the/pose/folder>")
        sys.exit(1)

    file_path = sys.argv[1]
    pose_folder_path = sys.argv[2]
    file_content = None

    try:
        with open(file_path, 'r') as file:
            file_content = file.read()
    except FileNotFoundError:
        print(f"The file {file_path} was not found.")
        sys.exit(1)
    except Exception as e:
        print(f"An error occurred: {e}")
        sys.exit(1)

    poses = file_content.split('\n')[:-1]
    
    # Iterate through each pose and apply the transformation
    for pose in poses:
        image_id, obj_id, qw, qx, qy, qz, x, y, z = pose.split(',')
        Q = np.array([qw, qx, qy, qz], dtype=np.float32)
        translation = np.array([x, y, z], dtype=np.float32)
        matrix = quaternion_to_matrix(Q, translation)
        np.save(pose_folder_path + '/pose' + str(int(image_id)), matrix)

    print(f"Number of transformation processed: {len(poses)}")
\end{lstlisting}

%\cleardoublepage{}
\bigskip

\begin{lstlisting}[language=Python, label=lst:4, caption=Python script \texttt{invert\_mask.py} to invert the masks from a specified folder. We aim to have a black object set against a white background.]
"""
Author:      Jeremy Chaverot
Date:        December 10, 2023
Description: Invert the masks from a given folder.
"""

import cv2
import os
import sys


def inverse_masks_in_folder(folder_path):
	# Iterate through the list of files at the specified path
    for filename in os.listdir(folder_path):
    	# Filter to include only png image files and exclude MacOS temporary files
        if filename.endswith(".png") and not filename.startswith('._'):
            mask_path = os.path.join(folder_path, filename)
            try:
                # Read the mask image
                mask = cv2.imread(mask_path, cv2.IMREAD_GRAYSCALE)
                if mask is None:
                    print(f"Failed to read image: {mask_path}")
                    continue

                # Invert the mask
                inverted_mask = cv2.bitwise_not(mask)

                # Save the inverted mask with a temporary name
                temp_path = os.path.join(folder_path, "temp_" + filename)
                cv2.imwrite(temp_path, inverted_mask)

                # Delete the original mask
                os.remove(mask_path)

                # Rename the inverted mask to the original filename
                os.rename(temp_path, mask_path)
                print(f"Inverted and replaced mask for: {mask_path}")
            except Exception as e:
                print(f"Error processing {mask_path}: {e}")


if __name__ == "__main__":

	# Check if the correct number of arguments is provided
    if len(sys.argv) != 2:
        print("Usage: python invert_mask.py <folder_path>")
        sys.exit(1)

    folder_path = sys.argv[1]
    inverse_masks_in_folder(folder_path)
\end{lstlisting}

%\cleardoublepage{}
\bigskip

\begin{lstlisting}[language=Python, label=lst:5, caption=Python script \texttt{resize.py} designed to alter an image's size with respect to a specified resize factor.]
"""
Author:      Jeremy Chaverot
Date:        January 01, 2024
Description: Resize the images from a given folder.
"""

import os
import sys
from PIL import Image


def resize_images(folder_path, resize_factor):
	# Iterate through the list of files at the specified path
    for filename in os.listdir(folder_path):
    	# Filter to include only png image files and exclude MacOS temporary files
        if filename.endswith(".png") and not filename.startswith('._'):
            img_path = os.path.join(folder_path, filename)
            with Image.open(img_path) as img:
                # Calculate new size
                new_size = tuple([int(dim / resize_factor) for dim in img.size])
                # Resize the image
                resized_img = img.resize(new_size, Image.ANTIALIAS)
                # Save the resized image with a different name temporarily
                temp_path = os.path.join(folder_path, "temp_" + filename)
                resized_img.save(temp_path)

            # Delete the original image
            os.remove(img_path)

            # Rename the resized image to the original filename
            os.rename(temp_path, img_path)


if __name__ == "__main__":

	# Check if the correct number of arguments is provided
    if len(sys.argv) != 3:
        print("Usage: resize.py <folder_path> <resize_factor>")
        sys.exit(1)

    folder_path = sys.argv[1]
    factor = int(sys.argv[2])

    resize_images(folder_path, factor)
\end{lstlisting}

%\cleardoublepage{}

%\begin{lstlisting}[language=Python, label=lst:6, caption=Python script \texttt{.py}.]

%\end{lstlisting}

\cleardoublepage{}
