\chapter{Python Scripts}
\fancyhead[C]{\small\textsc{A. Python Scripts}}


\begin{lstlisting}[style=pythonstyle, label=lst:1, caption=Python code of the data loading class \texttt{SpaceCraftCVLabDatabase()} from file \texttt{database.py}.]
SPACECRAFT_ROOT='data/SpaceCraft'
class SpaceCraftCVLabDatabase(BaseDatabase):
    def __init__(self, database_name):
        super().__init__(database_name)
        _, self.model_name = database_name.split('/')
        self.img_ids = [str(k) for k in range(len(os.listdir(f'{SPACECRAFT_ROOT}/{self.model_name}/images')))]
        self.model = self.get_ply_model().astype(np.float32)
        self.object_center = np.zeros(3, dtype=np.float32)
        self.object_vert = np.asarray([0,0,1], np.float32)
        self.img_id2depth_range = {}
        self.img_id2pose = {}
        _, self.database_type = self.model_name.split('-')
        # Track 1: SpaceCraft dataset camera intrinsic matrix
        self.K=np.array([[1406.708374, 0.000000, 512.000000], [0.000000, 1406.708374, 512.000000], [0.000000, 0.000000, 1.000000]], dtype=np.float32)
        # Track 2: SwissCube dataset camera intrinsic matrix
        # self.K=np.array([[607.5730322397038, 0.000000, 512.000000], [0.000000, 607.5730322397038, 512.000000], [0.000000, 0.000000, 1.000000]], dtype=np.float32)
        if self.database_type == 'test':
            resize_factor = np.loadtxt(f"{SPACECRAFT_ROOT}/{self.model_name}/resize_factor.txt")
            self.K *= resize_factor
            self.K[2, 2] = 1
        # Note : This following portion of code is to uncomment when the ground truth poses are store in a JSON file,
        #        like for the SwissCube dataset.
        # with open(f'{SPACECRAFT_ROOT}/{self.model_name}/scene_gt.json', 'r') as file:
        #     self.pose = json.load(file)

    def get_ply_model(self):
        fn = Path(f'{SPACECRAFT_ROOT}/{self.model_name}/{self.model_name}.pkl')
        if fn.exists(): return read_pickle(str(fn))
        ply = plyfile.PlyData.read(f'{SPACECRAFT_ROOT}/{self.model_name}/{self.model_name}.ply')
        data = ply.elements[0].data
        x = data['x']
        y = data['y']
        z = data['z']
        model = np.stack([x, y, z], axis=-1)
        if model.shape[0]>4096:
            idxs = np.arange(model.shape[0])
            np.random.shuffle(idxs)
            model = model[idxs[:4096]]
        save_pickle(model, str(fn))
        return model

    def get_image(self, img_id):
        # Track 1: SpaceCraft dataset image format
        return imread(f'{SPACECRAFT_ROOT}/{self.model_name}/images/{int(img_id):04}.png')
        # Track 2: SwissCube dataset image format
        # return imread(f'{SPACECRAFT_ROOT}/{self.model_name}/images/{int(img_id):06}.jpg')

    def get_K(self, img_id):
        return np.copy(self.K)

    def get_pose(self, img_id):
        # Track 1: Gen6D's pose format
        if img_id in self.img_id2pose:
            return self.img_id2pose[img_id]
        else:
            pose = np.load(f'{SPACECRAFT_ROOT}/{self.model_name}/pose/pose{int(img_id)}.npy')
            self.img_id2pose[img_id] = pose
            return pose
        # Note : This following portion of code is to uncomment when the ground truth poses are store in a JSON file,
        #        like for the SwissCube dataset. Do not forget to comment track 1 above.
        # Track 2: SwissCube dataset json pose format
        # raw_pose = (self.pose[f'{int(img_id) + 1}'])[0]
        # rot_mat = np.asarray(raw_pose['cam_R_m2c'])
        # rot_mat = np.reshape(rot_mat, (3, 3))
        # transl_vec = np.asarray(raw_pose['cam_t_m2c'])
        # transl_vec = np.reshape(transl_vec, (3, 1))
        # pose = np.concatenate((rot_mat, transl_vec), axis=1)
        # return pose

    def get_img_ids(self):
        return self.img_ids.copy()

    def get_mask(self, img_id):
        # Track 1: SpaceCraft dataset image format
        return np.sum(imread(f'{SPACECRAFT_ROOT}/{self.model_name}/mask/{int(img_id):04}.png'),-1)>0
        # Track 2: SwissCube dataset image format
        #return np.sum(imread(f'{SPACECRAFT_ROOT}/{self.model_name}/mask/{int(img_id):06}.png'),-1)>0
\end{lstlisting}

\cleardoublepage{}

\begin{lstlisting}[style=pythonstyle, label=lst:2, caption=Python script \texttt{format.py} to randomly generate the training set and the test set based on a specified probability. This script is used when creating both the reference set and the query set from a common set of images. Should be run from Gen6D's root folder.]
"""
Author:      Jeremy Chaverot
Date:        November 29, 2023
Description: Create the files val.txt, train.txt and test.txt according to a test percentage.
"""

import os
import sys
import random


if __name__ == "__main__":

	# Check if the correct number of arguments is provided
    if len(sys.argv) != 3:
        print("Usage: python format.py <object_name> <test_percentage>")
        sys.exit(1)

    object = sys.argv[1]
    test_percentage = float(sys.argv[2])

    if (test_percentage < 0 or 1 < test_percentage):
        print("Wrong value for the variable <test_percentage>. Should be between 0 and 1 included.")
        sys.exit(1)

    # List of files in the given folder
    all_files = os.listdir(f'data/SpaceCraft/{object}/images')

    # Filter the list to select only images and exclude MacOS temporary files
    image_files = [file for file in all_files if file.lower().endswith(('.jpg')) and not file.startswith('._')]

    num_images = len(image_files)

    # Transform each image
    with open(f'data/SpaceCraft/{object}/train.txt', 'w') as train, open(f'data/SpaceCraft/{object}/test.txt', 'w') as test:
        for image_file in image_files:
            rand = random.random()
            image_path = 'SpaceCraft/hubble/images/' + image_file
            if (rand < test_percentage):
                test.write(image_path + '\n')
            else: train.write(image_path + '\n')

    print(f"Done splitting {num_images} images in train.txt and test.txt")
\end{lstlisting}

\cleardoublepage{}

\begin{lstlisting}[style=pythonstyle, label=lst:3, caption=Python script \texttt{to\_jpg.py} to transform every images of a specified folder into \texttt{jpg} format.]
"""
Author:      Jeremy Chaverot
Date:        November 20, 2023
Description: Transform every images of a folder into jpg format.
"""

import os
import sys
from PIL import Image


def transform_image(image_path):
    img = Image.open(image_path)
    new_image_path = image_path.split('.')[0] + '.jpg'
    img.save(new_image_path)


if __name__ == "__main__":

	# Check if the correct number of arguments is provided
    if len(sys.argv) != 2:
        print("Usage: python to_jpg.py </path/to/your/images>")
        sys.exit(1)

    folder_path = sys.argv[1]

    # List of files in the given folder
    all_files = os.listdir(folder_path)
    
    # Filter the list to select only images and exclude MacOS temporary files
    image_files = [file for file in all_files if file.lower().endswith(('.png', '.jpg', '.jpeg', '.gif', '.bmp')) and not file.startswith('._')]
    
    num_images = len(image_files)

    # Transform each image
    for image_file in image_files:
        image_path = os.path.join(folder_path, image_file)
        transform_image(image_path)
        os.remove(image_path)

    print(f"Number of images transformed into .jpg: {num_images}")
\end{lstlisting}


\cleardoublepage{}

\begin{lstlisting}[style=pythonstyle, label=lst:4, caption=Python script \texttt{quaternion\_to\_matrix.py} to transform a txt file with quaternions and the translation vector into multiple npy files containing the rotation matrix augmented with the translation vector.]
"""
Author:      Jeremy Chaverot
Date:        November 20, 2023
Description: Transform a txt file with quaternions and the translation vector into multiple npy files containing the rotation matrix concatenated with the translation vector.
"""

import numpy as np
import sys
import os


def quaternion_to_matrix(Q, translation):
    """
        Covert a quaternion and translation into a full three-dimensional augmented rotation matrix.

        Input
        :param Q: A 4 element array representing the quaternion (q0, q1, q2, q3).
        :param translation: A 3 element array representing the translation (x, y, z).

        Output
        :return: A 3x4 element matrix representing the 3D rotation matrix concatenated with the translation vector.
    """

    # Extract the values from arguments
    q0 = Q[0]
    q1 = Q[1]
    q2 = Q[2]
    q3 = Q[3]

    x = translation[0]
    y = translation[1]
    z = translation[2]

    # Compute the rotation matrix
    r00 = 2 * (q0 * q0 + q1 * q1) - 1
    r01 = 2 * (q1 * q2 - q0 * q3)
    r02 = 2 * (q1 * q3 + q0 * q2)

    r10 = 2 * (q1 * q2 + q0 * q3)
    r11 = 2 * (q0 * q0 + q2 * q2) - 1
    r12 = 2 * (q2 * q3 - q0 * q1)

    r20 = 2 * (q1 * q3 - q0 * q2)
    r21 = 2 * (q2 * q3 + q0 * q1)
    r22 = 2 * (q0 * q0 + q3 * q3) - 1

    # 3x3 rotation matrix concatenated with the 3x1 translation vector
    matrix = np.array([[r00, r01, r02, x],
                       [r10, r11, r12, y],
                       [r20, r21, r22, z]])

    return matrix


if __name__ == "__main__":

    # Check if the correct number of arguments is provided
    if len(sys.argv) != 3:
        print("Usage: python quaternion_to_matrix.py </path/to/your/text/file> </path/to/the/pose/folder>")
        sys.exit(1)

    file_path = sys.argv[1]
    pose_folder_path = sys.argv[2]
    file_content = None

    try:
        with open(file_path, 'r') as file:
            file_content = file.read()
    except FileNotFoundError:
        print(f"The file {file_path} was not found.")
        sys.exit(1)
    except Exception as e:
        print(f"An error occurred: {e}")
        sys.exit(1)

    poses = file_content.split('\n')[:-1]

    # Iterate through each pose, apply the transformation and save it
    for pose in poses:
        image_id, obj_id, q0, q1, q2, q3, x, y, z = pose.split(',')
        Q = np.array([q0, q1, q2, q3], dtype=np.float32)
        translation = np.array([x, y, z], dtype=np.float32)
        matrix = quaternion_to_matrix(Q, translation)
        np.save(pose_folder_path + '/pose' + str(int(image_id)), matrix)

    print(f"Number of transformation processed: {len(poses)}.")
\end{lstlisting}

\cleardoublepage{}

\begin{lstlisting}[style=pythonstyle, label=lst:5, caption=Python script \texttt{invert\_mask.py} to invert the masks from a specified folder. We aim to have a black object set against a white background.]
"""
Author:      Jeremy Chaverot
Date:        December 10, 2023
Description: Invert the masks from a given folder.
"""

import cv2
import os
import sys


def inverse_masks_in_folder(folder_path):
	# Iterate through the list of files at the specified path
    for filename in os.listdir(folder_path):
    	# Filter to include only png image files and exclude MacOS temporary files
        if filename.endswith(".png") and not filename.startswith('._'):
            mask_path = os.path.join(folder_path, filename)
            try:
                
                mask = cv2.imread(mask_path, cv2.IMREAD_GRAYSCALE)
                if mask is None:
                    print(f"Failed to read image: {mask_path}")
                    continue               
                inverted_mask = cv2.bitwise_not(mask)    
                temp_path = os.path.join(folder_path, "temp_" + filename)
                cv2.imwrite(temp_path, inverted_mask)

                # Delete the original wrong mask
                os.remove(mask_path)

                # Rename the inverted mask
                os.rename(temp_path, mask_path)
                print(f"Inverted and replaced mask for: {mask_path}")
            except Exception as e:
                print(f"Error processing {mask_path}: {e}")


if __name__ == "__main__":

	# Check if the correct number of arguments is provided
    if len(sys.argv) != 2:
        print("Usage: python invert_mask.py <folder_path>")
        sys.exit(1)

    folder_path = sys.argv[1]
    inverse_masks_in_folder(folder_path)
\end{lstlisting}

\cleardoublepage{}

\begin{lstlisting}[style=pythonstyle, label=lst:6, caption=Python script \texttt{resize.py} designed to alter an image's size with respect to a specified resize factor.]
"""
Author:      Jeremy Chaverot
Date:        January 01, 2024
Description: Resize the images from a given folder.
"""

import os
import sys
from PIL import Image


def resize_images(folder_path, resize_factor):
	# Iterate through the list of files at the specified path
    for filename in os.listdir(folder_path):
    	# Filter to include only png image files and exclude MacOS temporary files
        if filename.endswith(".png") and not filename.startswith('._'):
            img_path = os.path.join(folder_path, filename)
            with Image.open(img_path) as img:
                new_size = tuple([int(dim * resize_factor) for dim in img.size])
                resized_img = img.resize(new_size, Image.ANTIALIAS)
                temp_path = os.path.join(folder_path, "temp_" + filename)
                resized_img.save(temp_path)

            # Delete the original image
            os.remove(img_path)

            # Rename the resized image
            os.rename(temp_path, img_path)


if __name__ == "__main__":

	# Check if the correct number of arguments is provided
    if len(sys.argv) != 3:
        print("Usage: resize.py <folder_path> <resize_factor>")
        sys.exit(1)

    folder_path = sys.argv[1]
    factor = int(sys.argv[2])

    resize_images(folder_path, factor)
\end{lstlisting}

%\cleardoublepage{}

%\begin{lstlisting}[style=pythonstyle, label=lst:6, caption=Python script \texttt{.py}.]

%\end{lstlisting}

\cleardoublepage{}
